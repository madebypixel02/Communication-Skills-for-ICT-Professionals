% Options for packages loaded elsewhere
% Options for packages loaded elsewhere
\PassOptionsToPackage{unicode}{hyperref}
\PassOptionsToPackage{hyphens}{url}
\PassOptionsToPackage{dvipsnames,svgnames,x11names}{xcolor}
%
\documentclass[
]{report}
\usepackage{xcolor}
\usepackage[top=30mm,left=20mm,heightrounded]{geometry}
\usepackage{amsmath,amssymb}
\setcounter{secnumdepth}{-\maxdimen} % remove section numbering
\usepackage{iftex}
\ifPDFTeX
  \usepackage[T1]{fontenc}
  \usepackage[utf8]{inputenc}
  \usepackage{textcomp} % provide euro and other symbols
\else % if luatex or xetex
  \usepackage{unicode-math} % this also loads fontspec
  \defaultfontfeatures{Scale=MatchLowercase}
  \defaultfontfeatures[\rmfamily]{Ligatures=TeX,Scale=1}
\fi
\usepackage{lmodern}
\ifPDFTeX\else
  % xetex/luatex font selection
\fi
% Use upquote if available, for straight quotes in verbatim environments
\IfFileExists{upquote.sty}{\usepackage{upquote}}{}
\IfFileExists{microtype.sty}{% use microtype if available
  \usepackage[]{microtype}
  \UseMicrotypeSet[protrusion]{basicmath} % disable protrusion for tt fonts
}{}
\usepackage{setspace}
\makeatletter
\@ifundefined{KOMAClassName}{% if non-KOMA class
  \IfFileExists{parskip.sty}{%
    \usepackage{parskip}
  }{% else
    \setlength{\parindent}{0pt}
    \setlength{\parskip}{6pt plus 2pt minus 1pt}}
}{% if KOMA class
  \KOMAoptions{parskip=half}}
\makeatother
% Make \paragraph and \subparagraph free-standing
\makeatletter
\ifx\paragraph\undefined\else
  \let\oldparagraph\paragraph
  \renewcommand{\paragraph}{
    \@ifstar
      \xxxParagraphStar
      \xxxParagraphNoStar
  }
  \newcommand{\xxxParagraphStar}[1]{\oldparagraph*{#1}\mbox{}}
  \newcommand{\xxxParagraphNoStar}[1]{\oldparagraph{#1}\mbox{}}
\fi
\ifx\subparagraph\undefined\else
  \let\oldsubparagraph\subparagraph
  \renewcommand{\subparagraph}{
    \@ifstar
      \xxxSubParagraphStar
      \xxxSubParagraphNoStar
  }
  \newcommand{\xxxSubParagraphStar}[1]{\oldsubparagraph*{#1}\mbox{}}
  \newcommand{\xxxSubParagraphNoStar}[1]{\oldsubparagraph{#1}\mbox{}}
\fi
\makeatother


\usepackage{longtable,booktabs,array}
\usepackage{calc} % for calculating minipage widths
% Correct order of tables after \paragraph or \subparagraph
\usepackage{etoolbox}
\makeatletter
\patchcmd\longtable{\par}{\if@noskipsec\mbox{}\fi\par}{}{}
\makeatother
% Allow footnotes in longtable head/foot
\IfFileExists{footnotehyper.sty}{\usepackage{footnotehyper}}{\usepackage{footnote}}
\makesavenoteenv{longtable}
\usepackage{graphicx}
\makeatletter
\newsavebox\pandoc@box
\newcommand*\pandocbounded[1]{% scales image to fit in text height/width
  \sbox\pandoc@box{#1}%
  \Gscale@div\@tempa{\textheight}{\dimexpr\ht\pandoc@box+\dp\pandoc@box\relax}%
  \Gscale@div\@tempb{\linewidth}{\wd\pandoc@box}%
  \ifdim\@tempb\p@<\@tempa\p@\let\@tempa\@tempb\fi% select the smaller of both
  \ifdim\@tempa\p@<\p@\scalebox{\@tempa}{\usebox\pandoc@box}%
  \else\usebox{\pandoc@box}%
  \fi%
}
% Set default figure placement to htbp
\def\fps@figure{htbp}
\makeatother





\setlength{\emergencystretch}{3em} % prevent overfull lines

\providecommand{\tightlist}{%
  \setlength{\itemsep}{0pt}\setlength{\parskip}{0pt}}



 


\usepackage{fancyhdr}
  \pagestyle{fancy}
  \fancyhead[C]{UOC}
  \fancyhead[R]{aperez-b@uoc.edu}
  \fancyfoot{}
  \fancyfoot[C]{\thepage}
\usepackage{fvextra}
\DefineVerbatimEnvironment{Highlighting}{Verbatim}{breaklines,commandchars=\\\{\}}
\DefineVerbatimEnvironment{OutputCode}{Verbatim}{breaklines,commandchars=\\\{\}}
\makeatletter
\@ifpackageloaded{caption}{}{\usepackage{caption}}
\AtBeginDocument{%
\ifdefined\contentsname
  \renewcommand*\contentsname{Table of contents}
\else
  \newcommand\contentsname{Table of contents}
\fi
\ifdefined\listfigurename
  \renewcommand*\listfigurename{List of Figures}
\else
  \newcommand\listfigurename{List of Figures}
\fi
\ifdefined\listtablename
  \renewcommand*\listtablename{List of Tables}
\else
  \newcommand\listtablename{List of Tables}
\fi
\ifdefined\figurename
  \renewcommand*\figurename{Figure}
\else
  \newcommand\figurename{Figure}
\fi
\ifdefined\tablename
  \renewcommand*\tablename{Table}
\else
  \newcommand\tablename{Table}
\fi
}
\@ifpackageloaded{float}{}{\usepackage{float}}
\floatstyle{ruled}
\@ifundefined{c@chapter}{\newfloat{codelisting}{h}{lop}}{\newfloat{codelisting}{h}{lop}[chapter]}
\floatname{codelisting}{Listing}
\newcommand*\listoflistings{\listof{codelisting}{List of Listings}}
\makeatother
\makeatletter
\makeatother
\makeatletter
\@ifpackageloaded{caption}{}{\usepackage{caption}}
\@ifpackageloaded{subcaption}{}{\usepackage{subcaption}}
\makeatother
\usepackage{bookmark}
\IfFileExists{xurl.sty}{\usepackage{xurl}}{} % add URL line breaks if available
\urlstyle{same}
\hypersetup{
  pdftitle={PR 4},
  pdfauthor={Alejandro Pérez Bueno},
  colorlinks=true,
  linkcolor={blue},
  filecolor={Maroon},
  citecolor={Blue},
  urlcolor={Blue},
  pdfcreator={LaTeX via pandoc}}


\title{PR 4}
\usepackage{etoolbox}
\makeatletter
\providecommand{\subtitle}[1]{% add subtitle to \maketitle
  \apptocmd{\@title}{\par {\large #1 \par}}{}{}
}
\makeatother
\subtitle{Digital Transformation Initiative for Medicat NGO:
Implementation of an Integrated Pharmaceutical Inventory Management
System}
\author{Alejandro Pérez Bueno}
\date{Jun 05, 2025}
\begin{document}
\maketitle

\renewcommand*\contentsname{Table of Contents}
{
\hypersetup{linkcolor=}
\setcounter{tocdepth}{2}
\tableofcontents
}

\setstretch{1.25}
\newpage{}

\chapter{Part I.Digital Transformation Initiative for Medicat
NGO}\label{part-i.digital-transformation-initiative-for-medicat-ngo}

\emph{Implementation of an Integrated Pharmaceutical Inventory
Management System}

\section{Executive Summary}\label{executive-summary}

InfoFarma is pleased to present this project proposal for the digital
transformation of Medicat, a Catalonia-based NGO dedicated to
distributing pharmaceutical products in developing African countries.
Founded in 1995, Medicat has relied on manual inventory tracking methods
despite exponential growth in both scope and pharmaceutical volume. The
proposed digital transformation aims to modernize operations, enhance
efficiency, and improve the NGO's capacity to fulfill its humanitarian
mission.The project will be implemented in three distinct phases. Phase
one will focus on developing a comprehensive digital catalog of
Medicat's pharmaceutical inventory across all warehouse locations. This
catalog will leverage the Spanish Agency for Medicines and Health
Products' (AEMPS) CIMA REST API, which provides standardized, detailed
information on pharmaceuticals available in Spain. By integrating with
this trusted data source, Medicat will ensure accurate and consistent
product information throughout its supply chain.

Phase two will deliver a web-based application with robust search
capabilities, enabling staff to monitor stock levels across all sites,
generate automated restock alerts, and produce accurate statistical
reports -- previously a manual and imprecise task critical for
government reporting requirements. Recognizing the prevalence of mobile
devices in Medicat's operational areas, this phase will also include a
mobile application mirroring the web platform's core functionality,
downloadable from Google Play and designed to operate effectively in
areas with limited network connectivity.

Phase three will enhance the end-user experience by introducing
additional mobile features to streamline medication distribution. These
features will allow patients to register digitally, enable doctors to
assign medications electronically, and empower supply teams to prepare
orders in advance -- significantly reducing the currently long wait
times at Medicat distribution sites.

The proposed solution addresses Medicat's immediate inventory management
challenges while establishing a scalable foundation for future digital
initiatives. By transforming manual processes into efficient digital
workflows, InfoFarma will help Medicat improve data accuracy, enhance
decision-making capabilities, optimize resource allocation, and
ultimately increase the impact of its humanitarian efforts in
resource-constrained environments.

\section{Introduction (revised from
PR3)}\label{introduction-revised-from-pr3}

Medicat, a Catalonia-based NGO established in 1995, has built a
commendable reputation for its humanitarian efforts in developing
African countries through the distribution of pharmaceutical products.
Operating with a lean team of professionals, the organization has relied
on the generous contributions of pharmaceutical laboratories and private
donors to build its substantial stockpile of medical supplies. Since its
inception, Medicat has employed manual inventory tracking methods, a
system that adequately served its needs during the organization's
formative years. However, the exponential growth in both the NGO's
operational scope and the volume of pharmaceuticals it manages has
rendered these traditional methods increasingly inadequate, creating
operational inefficiencies that potentially limit the organization's
humanitarian impact.The current manual inventory system presents several
critical challenges for Medicat's operations. Warehouse staff must
physically count and record stock levels across multiple storage
locations, a time-consuming process prone to human error. The absence of
real-time inventory visibility often results in suboptimal stock
distribution, with some locations experiencing shortages while others
maintain excess inventory of the same items. Moreover, the manual
compilation of data for government reporting requirements is laborious
and frequently imprecise, potentially affecting the organization's
compliance standing and funding opportunities. These operational
inefficiencies are particularly concerning given Medicat's mission to
provide life-saving medications to vulnerable populations in
resource-constrained environments, where any delays or inaccuracies in
inventory management can have significant humanitarian consequences.

In African countries where Medicat operates, pharmaceutical distribution
faces additional unique challenges. Limited network infrastructure in
remote areas complicates real-time data synchronization, while
inconsistent power supply affects the reliability of
technology-dependent systems. Furthermore, medication distribution sites
often experience long queues as staff manually verify prescriptions and
source supplies from warehouses. These region-specific challenges
necessitate a tailored digital solution that can function effectively
within these constraints while addressing Medicat's core inventory
management needs. The proposed digital transformation must therefore be
designed with these operational realities in mind, incorporating offline
functionality and mobile accessibility to ensure usability across all
Medicat's distribution sites, regardless of their technological
infrastructure.

The integration of the Spanish Agency for Medicines and Health Products'
(AEMPS) CIMA REST API presents a significant opportunity to enhance the
quality and consistency of Medicat's pharmaceutical data. This
comprehensive database, accessible through open-data protocols, contains
standardized information on medications available in Spain, including
identification details, technical specifications, authorization status,
and supply availability. By leveraging this resource, Medicat can
establish a reliable foundation of pharmaceutical information that
ensures accuracy across its inventory system. The CIMA database also
provides access to critical documentation such as package leaflets and
technical fact sheets, which can support staff training and enhance the
safe distribution of medications. This integration represents a
cost-effective approach to establishing a robust pharmaceutical
information system, building upon existing public resources rather than
developing a proprietary database from scratch.

The widespread adoption of mobile devices in Medicat's operational
regions offers a promising avenue for technological intervention.
Despite limited network infrastructure, mobile phone penetration in many
African countries has grown substantially, with devices becoming
increasingly accessible to both healthcare professionals and patients.
This technological reality presents an opportunity to implement a
mobile-first approach that aligns with existing resource availability. A
mobile application designed to operate with minimal network connectivity
can empower field staff to access and update inventory information,
verify pharmaceutical details, and process distribution requests even in
remote locations. Furthermore, the implementation of a patient-facing
mobile interface could significantly streamline the medication
distribution process, reducing wait times and improving the overall
service experience for beneficiaries.

The timing for this digital transformation initiative is particularly
opportune. Under the leadership of the new president, Julia de las
Heras, Medicat has demonstrated a clear commitment to modernizing its
operations to enhance its humanitarian impact. The organization's
recognition of its technological limitations and its proactive approach
to seeking external expertise signal an institutional readiness for
change. Simultaneously, advancements in cloud computing, mobile
technology, and offline-first application development have created a
favorable technological environment for implementing robust solutions in
resource-constrained settings. The maturation of the CIMA API also
provides a timely opportunity to integrate standardized pharmaceutical
data into Medicat's operations. By proceeding with this digital
transformation now, Medicat can position itself at the forefront of
technology-enabled humanitarian work, potentially establishing best
practices that could benefit similar organizations operating in
challenging environments.

\section{Project Description}\label{project-description}

\subsection{3.1. Project Objectives}\label{project-objectives}

\subsubsection{3.1.1. General Objective}\label{general-objective}

To implement a comprehensive digital inventory management system for
Medicat that modernizes pharmaceutical tracking processes, integrates
with the CIMA database, and enhances service delivery across all
distribution sites in developing African countries.

\subsubsection{3.1.2. Specific Objectives}\label{specific-objectives}

\begin{itemize}
\item
  \textbf{Establish a centralized digital catalog} utilizing the CIMA
  REST API to create a comprehensive database of pharmaceutical products
  with standardized information including technical specifications,
  authorization status, and supply data.
\item
  \textbf{Develop a real-time inventory tracking system} across all
  Medicat warehouse locations to provide accurate stock level
  monitoring, automated restock alerts, and optimal resource allocation.
\item
  \textbf{Implement mobile-first technology solutions} that enable
  offline functionality for inventory management and distribution
  processes in areas with limited network connectivity.
\item
  \textbf{Create automated reporting capabilities} that generate
  accurate statistical reports for government compliance requirements,
  reducing manual compilation errors and enhancing organizational
  accountability.
\item
  \textbf{Streamline patient service delivery} through digital
  registration systems and electronic prescription management to reduce
  wait times and improve the overall experience at distribution sites.
\item
  \textbf{Ensure scalable and secure data management} that accommodates
  future growth while maintaining data integrity and protecting
  sensitive patient and inventory information.
\end{itemize}

\subsection{3.2. Expected Results}\label{expected-results}

Upon project completion, Medicat will possess a fully functional digital
inventory management system comprising three integrated technological
components. The first component will be a comprehensive pharmaceutical
database built upon the CIMA API foundation, providing staff with
reliable access to standardized medication information, technical
specifications, and regulatory data. This database will eliminate the
inconsistencies and gaps that currently characterize Medicat's manual
record-keeping system, ensuring that all personnel work with accurate
and up-to-date pharmaceutical information.

The second component will be a web-based inventory management platform
featuring real-time stock monitoring across all warehouse locations.
This platform will enable administrators to track inventory levels,
identify supply shortages before they become critical, and generate
automated restock alerts based on predefined thresholds. The system will
include a robust reporting module that produces accurate statistical
analyses for government compliance requirements, eliminating the
labor-intensive manual compilation process that has historically
characterized this task. The platform will also provide data
visualization tools to help administrators identify trends, optimize
distribution patterns, and make data-driven decisions about resource
allocation.

The third component will be a mobile application designed specifically
for the operational realities of developing African countries. This
application will mirror the core functionality of the web platform while
incorporating offline capabilities that ensure continued operation
during network disruptions. The mobile app will enable field staff to
access inventory information, update stock levels, and verify
pharmaceutical details even in remote locations with limited
connectivity. Additionally, the application will include patient-facing
features that allow digital registration, electronic prescription
management, and pre-order capabilities, significantly reducing wait
times at distribution sites.

Beyond these technological deliverables, the project will result in
improved operational efficiency across all aspects of Medicat's
pharmaceutical distribution activities. Staff training programs will
ensure effective utilization of the new systems, while comprehensive
documentation will support ongoing maintenance and future enhancements.
The implementation will also establish best practices for digital
inventory management in resource-constrained environments, potentially
serving as a model for similar humanitarian organizations.

\subsection{3.3. Risk Analysis}\label{risk-analysis}

The implementation of Medicat's digital transformation initiative
presents several categories of risk that require proactive management
strategies. Technology-related risks constitute the primary concern,
particularly given the challenging operational environment in developing
African countries. Network connectivity limitations could potentially
disrupt real-time synchronization between the central database and field
locations. To mitigate this risk, the system architecture will
prioritize offline functionality, enabling continued operation during
network outages while automatically synchronizing data once connectivity
is restored. Additionally, backup power solutions and data redundancy
measures will be implemented to ensure system resilience against
infrastructure challenges.

User adoption represents another significant risk category, as staff
accustomed to manual processes may initially resist digital workflows.
Historical experience suggests that successful technology adoption
requires comprehensive change management, including thorough training
programs, ongoing support, and clear communication about the benefits of
the new system. To address this risk, the implementation will include
phased rollouts that allow gradual adaptation, user feedback
incorporation, and peer-to-peer learning opportunities. Champions within
the organization will be identified and trained as super-users to
provide ongoing support and encourage adoption among their colleagues.

Data security and privacy concerns require careful attention,
particularly given the sensitive nature of pharmaceutical inventory
information and patient data. The risk of unauthorized access or data
breaches could potentially compromise both Medicat's operations and
patient confidentiality. Comprehensive security measures will be
implemented, including encrypted data transmission, secure
authentication protocols, role-based access controls, and regular
security audits. Staff training will also emphasize data protection best
practices and compliance requirements.

Financial risks associated with cost overruns or extended implementation
timelines could strain Medicat's limited resources. To manage these
risks, the project will employ agile development methodologies that
enable early value delivery and iterative improvements. Clear project
milestones, regular progress reviews, and contingency planning will help
maintain budget and timeline adherence. Additionally, the modular
implementation approach allows for adjustment of scope based on
available resources while ensuring that core functionality is delivered
within the agreed parameters.

Finally, integration challenges with existing systems and processes
present potential operational risks. The CIMA API integration must be
thoroughly tested to ensure reliable data access and quality.
Compatibility issues between new mobile applications and existing
devices could limit adoption. To mitigate these risks, extensive testing
protocols will be implemented throughout development, including user
acceptance testing with actual Medicat staff. Backup procedures will be
maintained for critical processes during the transition period, ensuring
that operations can continue if technical issues arise during
implementation.

\chapter{Part II: Learning
Reflection}\label{part-ii-learning-reflection}

\emph{A Journey Through Written Communication}

As I reach the conclusion of this semester-long journey in written
communication, I find myself compelled to examine the transformation
that has occurred in my writing abilities and professional perspective.
The assertion that improvement has taken place is not merely optimistic
speculation but rather a demonstrable reality that becomes evident when
comparing my initial submissions with the complexity and sophistication
of my current work. This improvement manifests most clearly in my
growing understanding of audience analysis, my enhanced ability to
structure arguments coherently, and my developing mastery of
professional tone and register.The most significant advancement in my
writing capabilities has occurred in the realm of content development
and argumentation. At the semester's outset, my approach to professional
writing was largely intuitive, relying on general communication
principles without fully understanding the nuanced requirements of
specific genres. The systematic study of project proposals, executive
summaries, and professional emails has provided me with concrete
frameworks for organizing information and presenting arguments
persuasively. I have learned to distinguish between different types of
evidence, to anticipate reader questions and objections, and to
structure my writing in ways that guide readers toward specific
conclusions. This growth is particularly evident in my evolution from
the descriptive approach of early assignments to the analytical and
strategic thinking demonstrated in my recent project proposal work.

Reflecting on the feedback received across the semester, I can see that
my strongest area has consistently been the use of appropriate register
and technical vocabulary, as highlighted by positive remarks on my
formal tone and precise language in PR1 and PR2; however, the greatest
challenge emerged in PR3, where despite a well-structured and coherent
text, my content relevance faltered by not aligning with the assigned
topic, revealing that while my control of language and structure is
solid, maintaining strict topical focus and fully addressing the
communicative task remains an area for ongoing improvement.

The feedback process implemented throughout this semester has proven
invaluable in accelerating my development as a professional writer. The
progressive nature of the rubric application, where punctuation errors
that were simply noted in PR1 became grade-affecting elements by PR3,
created a scaffolded learning environment that allowed me to focus on
different aspects of writing quality at appropriate stages of my
development. The detailed comments provided on each submission offered
specific, actionable guidance that enabled me to understand not merely
what needed improvement but why certain changes would enhance the
effectiveness of my communication. This approach proved far more
beneficial than generic feedback would have been, as it was tailored to
my specific writing patterns and areas of weakness.

Perhaps most significantly, the feedback process taught me to become a
more critical reader of my own work. The experience of receiving
detailed commentary on my writing choices has made me more conscious of
decisions I make regarding word choice, sentence structure, and
organizational patterns. I now approach revision not as a superficial
editing process but as an opportunity to fundamentally reconsider how
effectively my writing achieves its intended purpose. This shift in
mindset represents perhaps the most valuable outcome of the semester, as
it provides a foundation for continued improvement even after formal
instruction has concluded.

The practical application of these writing skills extends far beyond the
academic context of this course. In my future professional endeavors,
the ability to craft compelling project proposals, write persuasive
business correspondence, and develop clear technical documentation will
prove essential for career advancement and organizational effectiveness.
The specific experience of analyzing complex professional scenarios and
translating them into coherent written proposals has provided me with
tools that will be directly applicable in consulting environments,
project management roles, and any position requiring strategic
communication with stakeholders. Moreover, the critical thinking skills
developed through this process of analyzing audience needs,
organizational contexts, and communication objectives will enhance my
ability to solve problems and make decisions across various professional
domains.

\chapter{Part III: Video submission}\label{part-iii-video-submission}

\emph{Applying Professional Communication Skills}




\end{document}
